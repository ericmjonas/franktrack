\documentclass{article}
\usepackage[citestyle=numeric]{biblatex}
\usepackage{fullpage}
\addbibresource{library.bib}

\title{Real-Time Rodent Tracking}
\author{Eric Jonas}

\begin{document}
\maketitle

\begin{abstract}
Here I propose a short-duration project (four to six weeks) with two
goals: advance the state-of-the-art in rodent behavioral tracking for
awake behaving electrophysiology and to expand my understanding of
sequential Monte Carlo methods.
\end{abstract}


\section{Motivation}
Sequential Monte Carlo (SMC) methods differ tremendously in both
mathematical foundation and practical application from the
Markov-Chain Monte Carlo methods that I have been using successfully
for the past five years. They excel in time-series models with
ever-expanding state-spaces and long-running correlations, but at the
same time provide weaker asymptotic guarantees than MCMC. Part of my
research/academic interest has been determining which inference
methods let me focus on building more complex models without worry
about inference details. I believe that some recent advances in SMC
methods might be getting us close to that point, but this is
based on a cursory review of the literature and a very limited
background.

Simultaneously, it has been a very long time since I've looked at
non-econometric time series, and I've never tackled a tracking problem
before. I miss working with real data in a neural context, and if I'm
going to attempt to contribute going forward to the field of neural
data analysis, I'm going to need more experience. And I'm on vacation,
and this is a lot more fun than going to the beach.

\section{Proposal}

Currently animals are tracked using software that implements various
heuristics and, in regions of uncertainty, asks for human
intervention. The utility of this approach is the resulting tracking
data is very ``clean'' for downstream analysis, with the downside that
in complex environments a great deal of human annotation is necessary.

I propose developing a state-space model that explicitly models the
animal's location, orientation, and other kinematic properties to
enable high-fidelity reconstruction of the animal's position. I'm
breaking this down into three components:

\begin{itemize}
\item \textbf{Phase 1} : sequential state estimation -- develop various
  models for estimating the animal's position $x_t$ at time $t$ (the
  ``filtering'' problem). Implement and evaluate those models across
  existing data sets.
\item \textbf{Phase 2} : smoothing -- higher quality results should be
  possible exploiting data from $1..T$ to estimate $x_t$. Smoothing
  algorithms are more suited for offline analysis, but are generally
  more complex and have higher memory requirements.
\item \textbf{Phase 3} : online / real-time tracking -- implement the
  filtering algorithm with constant-memory and constant-runtime
  performance in a manner suitable for use in real-time applications,
  and optionally extend with more real-time sensors (such as
  accelerometers and gyroscopes) for final temporal accuracy.
\end{itemize}

The existing diode-tracking system has several challenges which we
will account for, and will produce an exciting model:

\begin{itemize}
\item \textbf{Occlusion} In many frames, it is hard to see both LEDs. They are
  differentiated by size, and when one is occluded it can be confusing
  to determine whether you are tracking the large or small one.
\item \textbf{Complex behavior} The animal will rear in various contexts,
  which means that the LED's motion is not simply in the $x-y$ plane.
\item \textbf{Reflection} The LEDs are quite bright relative to the
  background, and will sometimes reflect off nearby surfaces
\item \textbf{Correlations} A great deal of short-term occlusion comes from
  the wire bundle emerging from the head stage. Due to the geometry of
  the setup, then, occlusion will be strongly correlated with
  position.
\item \textbf{Diversity of experimental environment} Rodents are run in many
  different environments, with different geometries and different
  occupancy profiles.
\item \textbf{Long-running} Experiments can be many hours long, and at a
  framerate of 60 Hz that means over $200k$ points per hour. In
  these regimes, SMC methods can suffer from pathologies like particle
  depletion.
\item \textbf{Complex environmental motion} Often, other things in the scene
  can be moving, ranging from parts of the experiment (for reward
  delivery or track reconfiguration) to simply the experimenter.
\end{itemize}

\section{Methodology}
Sequential Monte Carlo methods, beginning with the rise of
Sequential-Importance-Resampling in 1993 \cite{Gordon1993}, have become
the go-to method for nonlinear time series state estimation
problems. They have seen a great deal of success in problems around
tracking and navigation, especially in the areas of simultaneous
localization and mapping \cite{FastSLAM2002}, where the goal of a robot is
to explore a novel space while simultaneously building up a map of
that space.

Recent advances hint at the possibility of these methods becoming
substantially more powerful for the average experimenter. This
includes the incorporation of various types of Markov-chain Monte
Carlo into SMC \cite{Gilks2001} and vice-versa \cite{Andrieu2010b},
automatic decomposition of state spaces for efficient inference
\cite{Ahmed2012}, and the development of exact approximations for
previously-analytic operations like Rao-Blackwellization
\cite{JohansenExact2012}. Over the course of this project I want to become an
expert in these contemporary methods, reimplementing many of them from
recent papers to evaluate their suitability in real-world tracking
problems.

My other love, nonparametric Bayesian methods, have seen increasing
use in time series analysis \cite{FoxThesis2009}. While outside of the
scope of this project, the hope is that the generative state-space
view encouraged by SMC will serve as a natural bridge to the more
familiar ground of np-Bayes.

\section{Data}

The input and evaluation data will be existing tracking datasets
provided by the Frank Lab. These consist of a video of the animal's
behavior along with the human-annotated tracking data. The more of
these I can obtain, the better, as they will give a better indication
of the variety of animal behaviors that can be expected. I will curate
these data into a collection of training sets for algorithmic
evaluation.


\section{Tools}

I plan to experiment, write, and deliver everything in Python. Python
is an easy-to-use interpreted programming language that has seen
widespread adoption by the same scientific community that has
traditionally relied on MATLAB for experimental data analysis. The
addition of various excellent libraries (Including numpy \cite{numpy}
for matrix-vector operations, SciPy \cite{scipy} for scientific
functions, and Matplotlib \cite{matplotlib} for plotting) have made it
an excellent matlab replacement. When Python's interpreted speed
becomes a bottleneck, it is now trivial to implement c-level functions
using Cython\cite{cython}.

The rise of ubiquitous cloud computing infrastructure has made it
possible for researchers to easily exploit computing resources across
thousands of machines. The per-cpu-core-hour pricing model results in
100 hours of computing time costing several dollars, and the
parallelization and elasticity afforded allow for those 100 hours to
be completed in one hour of wall-clock time. I plan on using
Pi-cloud's excellent services \cite{picloud} to test algorithm changes
across multiple datasets simultaneously.

\section{Deliverables}
All source code will be delivered in a git repository served on Github
\cite{github}, as well as documentation, pseudocode, and results. I
will write up the results in journal format and would like to
ultimately submit it to one of the IEEE journals on tracking or
time-series analytics.

The tracking code program will work similar to the existing program,
where an input mpeg video along with timing synchronization
information will yield a file consisting of position data. It's likely
that the output file will condition additional information not
provided by the current tracking software, including head direction
and a confidence bound around the tracking estimate.


\section{Conclusion}
The expected duration of the project is between four and six weeks,
with weekly meetings at the Frank Lab to review progress. Total
expenses for computing resources are estimated at less than
\$1000. Should this project succeed, it will hopefully save
researchers many hours of hand-labeling video and provide
higher-quality spatial data for subsequent analysis. The realtime
potential will hopefully contribute to current experiments at online
experiment design.

\printbibliography


\end{document}
